% Created 2025-07-14 Mon 10:41
% Intended LaTeX compiler: lualatex
\documentclass[11pt]{article}
\usepackage{graphicx}
\usepackage{longtable}
\usepackage{wrapfig}
\usepackage{rotating}
\usepackage[normalem]{ulem}
\usepackage{amsmath}
\usepackage{amssymb}
\usepackage{capt-of}
\usepackage{hyperref}
\author{rnouwen}
\date{\today}
\title{}
\hypersetup{
 pdfauthor={rnouwen},
 pdftitle={},
 pdfkeywords={},
 pdfsubject={},
 pdfcreator={Emacs 28.2 (Org mode 9.5.5)}, 
 pdflang={English}}
\begin{document}

\tableofcontents

---
title: Formal languages
author: Johannes Korbmacher \& Rick Nouwen
weight: 3
resources:
\begin{itemize}
\item src: img/inferences.png
name: inferences
\end{itemize}
params: 
  date: 09/09/2024
  last\textsubscript{edited}: 30/4/2025
  id: txt-lang
  math: true
---

In \{\{< chapter\textsubscript{ref} chapter="logic-and-ai" id="logical-systems">\}\}
Chapter 1. Logic and AI\{\{< /chapter\textsubscript{ref} >\}\} and \{\{< chapter\textsubscript{ref}
chapter="valid-inference" id="formalization">\}\}
Chapter 2. Valid inference\{\{< /chapter\textsubscript{ref} >\}\}, we introduced the idea of a formal language as a way of introducing mathematical precision to our approach to valid inference.

In this chapter, you'll learn the nitty-gritty of how formal languages are
defined mathematically, and we'll look at some of the \uline{many} applications of
formal languages in AI and beyond. 

We'll be discussing the following:

\begin{itemize}
\item What \uline{is} a [formal language] and how does it differ from a natural one?

\item We give a mathematical [definition](\#languages) of formal languages via [alphabets](\#alphabets) and
\end{itemize}
[grammars](\#grammar).

Once this definition is sufficiently clear, we'll move to:

\begin{itemize}
\item some [examples](\#examples)
\item and the idea reading or [parsing](\#parsing) a formal language.
\end{itemize}

We conclude with some [applications](\#applications).

\#\# Formal versus natural languages

\uline{English}, \uline{[St̓át̓imcets](\url{https://en.wikipedia.org/wiki/Lillooet\_language})} and \uline{[Ripuarian](\url{https://en.wikipedia.org/wiki/Ripuarian\_language})} are examples of natural languages. \uline{Python}, \uline{propositional logic} and \uline{[algebraic chess notation](\url{https://en.wikipedia.org/wiki/Algebraic\_notation\_(chess)})} are examples of formal languages. What makes a language a natural language and what makes it a formal one?

We all speak at least one natural language and many of us speak multiple. A natural language is a naturally evolved system that you learn spontaneously, for instance by interacting with your parents and other people around you when you are very young. Native speakers of English, St̓át̓imcets or Ripuarian didn't learn their native language at school or by studying grammar books, but simply by being in an environment where the language was used. Because natural languages are acquired in this way, they are also very susceptible to change. They constantly evolve, just by being used and passed on to next generations. 

In contrast, nobody learns python, propositional logic or algebraic chess notation simply by interacting with their parents. Also, these languages clearly didn't evolve naturally and while the conventions of these languages may change over time, they do not do so spontaneously, but rather because a community of users explicitly decides to make a certain change.  

Having command of a natural language is an extremely powerful thing. It allows you to communicate to others about your desires, your thoughts, your observations, your plans. It allows you to learn things in school, to teach other people what you have learned, to enjoy art in the form of literature, poetry and song lyrics, to laugh at jokes, to persuade others to change their actions, etc. etc. 

Most of the recent advances in AI make use of the fact that natural language is such a pervasive part of our lives. Because language is everywhere, it creates an enormous wealth of data about many facets of human existence and human cognition, ready as input for machine learning. Given this, why don't we just use natural language for everything in AI? What do we need formal languages for?

There are many different reasons formal languages are important in general and for AI in particular. One reason that is relatively quick to appreciate is that powerful large language models trained on natural language are developed using programming languages, which are formal languages. We cannot construct a neural network using natural language. So even sub-symbolic approaches to AI rely on formalisms that have symbolic roots. More generally, however, natural languages have some properties that make them unsuitable for doing logic or maths and, equally importantly, that make them unsuitable for storing human knowledge. Two such properties stand out:

\begin{itemize}
\item Natural languages are \uline{ambiguous}: Statements formulated in a natural language can often be interpreted in multiple ways. As a consequence, if we choose to use natural language as a basis for drawing inferences, we can't always be sure that rules or facts that we want would want an AI system to benefit from are understood in the appropriate way.

\item Natural languages are \uline{over-expressive}: Specific statements made in a natural language tend to describe highly specific thoughts. This makes natural language unsuitable for studying \uline{generalities} in valid inference.
\end{itemize}

Let us illustrate both these properties in more detail:

\#\#\# Natural language hosts ambiguity

Imagine that we want to build an AI system that gives out safety advice on eating foraged mushrooms. We have access to a lot of expert knowledge about mushrooms. One idea could be to feed this knowledge to the AI system in the form of natural language statements. For instance, we could give the system lots of English language sentences that together make up all our knowledge. Say, these sentences include the following:

> \(S_1\ =\) If a mushroom has red spots and gills, then it is not poisonous.

Also, we prompt the AI system with another English language sentence:

> \(S_2\ =\) The mushroom in front of me has red spots and gills.

It may seem straightforward how the AI system can prepare an advice on the basis of the knowledge captured in \(S_1\) and the information in the user prompt \(S_2\). We may think that all the AI system needs to do is recognise that it can apply modus ponens. From "If the door is open, you can come in" and "The door is open", you can infer that you can come in. Completely parallel then, you would expect that from the two statements above the AI system should infer that the mushroom in question is not poisonous. 

The problem, however, is that "X has red spots and gills" is ambiguous. It could either mean that X has red spots and red gills, or it could state that it has red spots and that it has gills (of whatever colour). Because of this we cannot be sure what either of these statements are saying exactly. It is not clear what the rule is that \(S_1\) is intended to capture, nor is it clear what observation the user is describing with \(S_2\). And because of all that uncertainty, we cannot be sure whether modus ponens applies. For instance, perhaps \(S_1\) is intended to mean that mushrooms that have red spots and red gills are not poisonous, while \(S_2\) is intended to mean that the mushroom in question has gills (gray ones, in fact) and red spots. In that case, modus ponens would not apply. In other words, if our AI system accidentally interprets these sentences not as they were intended, it could end up applying modus ponens and cause the users to poison themselves. 

A similar problem concerns words like "\textsubscript{if}\_" and "\textsubscript{then}\_" in languages like English. Say, I remove the ambiguity from \(S_1\) above and instead state that:

> \(S_1'\ =\) If a mushroom has red spots and red gills, then it is not poisonous.

It is now clear what this means. It tells us what is the case when a mushroom has the features that are mentioned. Does this tell us anything about mushrooms that do not have red spots and red gills? For most people, the intuition is that it does not: on the basis of just \(S_1'\) I cannot conclude anything about a mushroom with black gills and no spots. The problem, however, is that "\textsubscript{if}\_" and "\textsubscript{then}\_" are not always understood in this way. Imagine yourself saying \(S_3\) to a child:

> $$S_3\ =$$ If you behave well, I will buy you an ice-cream.

This clearly tells the child, via modus ponens, what happens when they are well-behaved. However, in this case the statement also seems to be saying what happens when they do not behave well. \(S_3\) seems to clearly suggest that if the child does \uline{not} behave well, then there won't be any icecream. So, "\textsubscript{if}\_" and "\textsubscript{then}\_" are interpreted differently in different examples. This again means that with natural language we often cannot be entirely sure what something means. 

Ambiguity is extremely common. For that reason it is important to use an unambiguous language for a theory of valid inference. More generally, whenever we want represent knowledge and rules precisely, we should avoid the inherent ambiguity of natural language. Formal languages allow us to do just that.

\#\#\# Over-expressiveness

To illustrate the problem of over-expressiveness, let us look at another case of modus ponens:

> \(M\) = If an object is placed in a normal cardboard box and subsequently nothing happens to that box, then the object will still be in the box.

Say a magician places a rabbit in a cardboard box and they close the box. After a short while they open it again and show the audience that the box is empty. The audience gasps. Why? Because on the basis of a common assumption like \(M\) and modus ponens, the audience expects the box to contain the rabbit. An object was placed in what looks like a normal box, we didn't see anything happening to the box, so we infer via modus ponens that the object is still in the box.

Members of the audience now need to somehow reconcile the empty box with what they saw. They have a number of options. It looked like the box was a normal box, but perhaps it wasn't. Perhaps there is some trick that lets the rabbit escape from the box unseen by the audience. In that case, this is not a normal box and modus ponens would not allow us to conclude that the rabbit is still in the box. Similarly, the audience didn't see anything happen to the box, but perhaps the magician managed to distract his audience and perhaps he removed the rabbit in a way we couldn't see. Once more, modus ponens does not apply and we do not end up inferring that the rabbit is in the box. Another possibility is that the rabbit is still in the box. That is, we were right to apply modus ponens, but we are wrong in our observation that the rabbit is gone. (Perhaps the magician isn't showing us all of the box?) Finally, and most interestingly, perhaps some people in the audience take this failure of modus ponens as evidence that \(M\) must be false. In other words, these people believe in magic. 

Now, compare this story to the following statement:

> > \(M'\): If you multiply an number with another number and both these numbers are odd, then the result will also be odd.

Say now I calculate \(a\times b=c\) and both \(a\) and \(b\) are odd. I observe that \(c\) is an even number, then there's a few options again. Perhaps I was wrong to believe that \(a\times b=c\), or perhaps I was wrong to believe that \(a\) and \(b\) are both odd. Or perhaps my observation that \(c\) is even was wrong. An arrogant person may perhaps even believe that their maths book is wrong in stating \(M'\). 

In any case, both the case of magic and the case of odd number multiplication show that modus ponens is a very strong inference. As soon as we have all the ingredients for modus ponens, we cannot help but draw the conclusion. And if that conclusion is not in line with what we observe, we start questioning our assumptions and our conclusion. 

The two stories also show that modus ponens is a very \uline{general} inference. It exists completely independent of subject matter. Structurally, the story above about \(M\) is completely parallel to that of \(M'\). This is where the over-expressiveness of natural language comes in. We can use natural language to express individual cases where modus ponens applies, as we just did above. But because natural language is so good at talking about specific details of situations, it is very bad at abstracting away. \(M\) and \(M'\) are highly specific examples of premises that with the right further premise bring us in a situation where we can apply modus ponens. To be able to talk about modus ponens as a \emph{general} principle of valid inference, we would need to let go of the specifics in these examples and state the principle using an abstract formal language.

In the formal language of propositional logic, the two stories can be captured in a single schema:

$$ A\rightarrow B,\ \ A\quad\models\quad B$$

\begin{center}
\begin{tabular}{lll}
 &  & \\
---- & ----- & -----\\
\(A\) & an object was placed in a normal box  and subsequently nothing happened to the box & two numbers were multiplied and both numbers were odd\\
\(B\) & the box is still empty & the result is odd\\
\(A\rightarrow B\) & \(M\) & \(M'\)\\
\end{tabular}
\end{center}

In both cases above, we thought \(A\rightarrow B\) was the case, as was \(A\). But in both cases, we thought \(B\) was not the case. Given that this clashes with modus ponens we start questioning our assumptions. Either something is wrong with our assumption \(A\) or \(A\rightarrow B\), or something is wrong with our belief that \(B\) is false. Logic allows us to do make such reasoning very explicit and very general. By using a formal language we can focus on the pattern underlying our mechanisms of valid inference.

In other words, when we study valid inference, we often do not care about the specific content of the statements our inferences are built on. It would be extremely hard to define a notion like modus ponens using just a natural language. The abstraction offered by a \uline{formal} language makes it possible to make explicit what all inferences that are to be classified as such have in common.

More generally, when we study systems of valid inference, we are often looking to find out what the consequences are of our assumptions about logical laws.  If we only had natural language to study this, which sentences should we use then? Should these be about mushrooms, rabbits, numbers, or ice creams? Similar considerations apply to mathematics. Highly abstract formal languages allow us to focus on the important things. We all learn that we can simplify a quadratic equations like $$x^2+5x+6=0$$ to $$(x+3)(x+2)=0$$, which means that $$x$$ is either -3 or -2. Just imagine doing this without the use of abstract symbols like $$x$$!

\#\# Formal languages for logic

In logical theory, the main role of formal languages is to provide a model of
\textbf{\textbf{logical form}}. As foreshadowed in \{\{< chapter\textsubscript{ref}
chapter="valid-inference" id="formalization">\}\}
Chapter 2. Valid inference\{\{< /chapter\textsubscript{ref} >\}\}, validity doesn't depend on
the concrete words or sentences involved. \uline{All} of the below inferences are
valid:

\begin{enumerate}
\item\relax [BERT](\url{https://en.wikipedia.org/wiki/BERT\_(language\_model)}) is either my
\end{enumerate}
favorite dog or an LLM, and BERT not an LLM. So, BERT is my favorite dog.

\begin{enumerate}
\item Given that Ada is either on the Philosopher’s Walk or in the study, and she’s
\end{enumerate}
not in the study, she therefore must be on the Philosopher’s Walk.

\begin{enumerate}
\item Either Johannes or [Data](\url{https://en.wikipedia.org/wiki/Data\_(Star\_Trek)}) is
\end{enumerate}
teaching this class and Data is not teaching it. So, Johannes is teaching the
class.

It's easy to recognize the pattern and generate more examples of valid
inferences like this. If we let \(A\) and \(B\) be arbitrary sentences, then the
pattern is $$A\text{ or }B, \text{ not B}\vDash A.$$
This is the logical form of inferences 1.-3., which, by the way, is known as
[disjunctive syllogism](\url{https://en.wikipedia.org/wiki/Disjunctive\_syllogism}). Using a formal language, we have overcome the problem of over-expressiveness: we can express this type of valid inference in a single formula.

From this perspective, formal languages are the result of introducing
\uline{placeholders} for the logically irrelevant parts of language and special
symbols, so-called \textbf{\textbf{logical constants}} for the logically relevant parts of
language.

In the case of disjunctive syllogism, for example, the relevant formal language
will use so-called \textbf{\textbf{sentence letters}} (\(p,q,r,\dots\)) to stand for arbitrary 
sentences, and \textbf{\textbf{propositional connectives}} for the logically relevant
[grammatical conjunctions](\url{https://en.wikipedia.org/wiki/Conjunction\_(grammar)})
(\(\neg\) for "not", \(\land\) for "and", \(\lor\) for "or", \(\rightarrow\) for "if \ldots{}, then
\ldots{}", \ldots{}). We get something like the following as the model for the shared logical form of inferences 1.-3. 

$$p\lor q,\neg p\vDash q$$ 

The resulting [logical
system](/textbook/logic-and-ai/\#logical-systems), known as \textbf{\textbf{propositional
logic}}, then, investigates the validity of these \uline{formal} inferences. Crucially, in that system each statement that can be expressed in not only entirely abstract, it is also unambiguous. In other words, each statement corresponds to a unique logical form.

Note that what's happening here is very much in line with the picture from \{\{< chapter\textsubscript{ref} chapter="logic-and-ai" id="logical-systems">\}\}
Chapter 1. Logic and AI\{\{< /chapter\textsubscript{ref} >\}\}, where we described logical systems
as \textbf{mathematical models} of valid inference. Mathematical models, remember, are
characterized by abstraction, idealization, and assumptions. Here it is easy to
see, for example, where the abstraction lies: we're abstracting away from
logically irrelevant features of language, such as which concrete sentence is
involved. A good example of idealization is that we're writing a single logical
constant, e.g. \(\land\), for the many different ways to express conjunction in
natural language: "and", "as well as", "together with", \ldots{}

From the perspective of logical theory, the main advantage of developing
mathematical models of language and, more concretely, logical form is that it
allows us to investigate valid inference with [**mathematical
rigor**](\url{https://en.wikipedia.org/wiki/Rigour\#Mathematics}), permitting us to
establish (meta-)logical facts \uline{beyond any reasonable doubt}. But in this
course, we take a slightly more pragmatic perspective at formal languages as a
\textbf{\textbf{logical tool}}.

For us, formal languages \textbf{\textbf{solve a fundamental problem}} for AI: \_How can we
store knowledge in such a way that we can communicate it to computational models
of intelligent behavior (computers)?\_ Formal languages solve this problem
because given their deterministic, mathematical nature, it is relatively easy to
teach them to computers. In fact, a fundamental insight from computer science is
that [programming
langauges](\url{https://en.wikipedia.org/wiki/Programming\_language}) are, for all
intents and purposes, formal languages. 

Moreover, for us as students of AI, it is important to note that [database
languages](\url{https://en.wikipedia.org/wiki/Database\#Database\_languages}), which are
used to create and search databases, [ontology
languages](\url{https://en.wikipedia.org/wiki/Ontology\_language}), which are used to
create representations of factual knowledge of the world, and so on \uline{all are}
formal languages. In short, in knowledge representation, logical methods reign
supreme.

\#\# Languages

So far, you only had a glimpse of what a formal language looks like. We have not specified one formally yet. Before we can go ahead and give the mathematical definition of what a formal
language is, we need to talk a bit more about \uline{sets}. Formal languages \uline{are}
sets. So, we need to know what a set is before we can talk about formal
languages.

In \{\{< chapter\textsubscript{ref} chapter="valid-inference" id="semantic-methods-for-deduction" >\}\}
2.3.1 \{\{< /chapter\textsubscript{ref} >\}\}, you first encountered sets. We now delve a bit
more into \textbf{\textbf{elementary set theory}}, the basic theory of sets.

A \uline{set} is the simplest kind of collection of objects. All that matters to a set is which things are in it and which things are not. If some object \(x\) is in a set, we say that \(x\) is one of its \textbf{\textbf{elements}} or \textbf{\textbf{members}}. Elements of a set are also said to \uline{belong to} the set or to \uline{be contained in} the set. Beyond membership, nothing matters to a set. For instance, there is no order to the elements in a set and an object is either in the set or not - it cannot be in a set multiple times. 

A set may contain any kind of object: numbers, symbols, people, or even other sets. For \(X\) a set and \(x\) an object, we write \(x\in X\) to say that \(x\) is an element of \(X\) and we write \(x \notin X\) to say that \(x\) is \uline{not} an element of \(X\). If we have many objects \(x_1, \ldots, x_n\), then we also write \(x_1, \ldots, x_n\in X\) to say that \(x_1\in X\), and \(x_2\in X\), and \(x_3\in X\), \(\ldots\), and \(x_n\in X\).

If the elements of a set are precisely \(a_1, \ldots, a_n\), then we can
denote the set by \$$\backslash${a\textsubscript{1}, \ldots, a\textsubscript{n}$\backslash$}.\$ This is called an
\textbf{\textbf{extensional definition}} of the set. So, the set \(\{1,a,
\{\text{Robbie},0\}\}\), for example, contains precisely the number 1, the
symbol \(a\), and the set \(\{\text{Robbie},0\}\), which in turn contains Robbie and the number 0 as elements.

For most interesting sets, however, we cannot give an extensional definition. One reason for this could be that we may not know exactly what the elements are. For instance, if the elements of a set are precisely the objects satisfying condition \(\Phi\), then we can denote the set by \$$\backslash${x:\(\Phi\)(x)$\backslash$}.\$ This is called a definition by \textbf{\textbf{set abstraction}}. For example, if I have the quadratic equation \(x^2+5x+6=0\), then we can express the set of solutions to this equation as:

$$\{x: x^2+5x+6=0\}$$

In other words, we have a way of expressing the solutions, even if we do not yet know what they are. (It turns out that this abstracted set is equal to the extensional set \(\{-3,-2\}\).)

Another reason why non-extensional definitions are handy is because many of the kinds of sets we want to study are typically infinite. 
 For example, \(\{x:x\text{ is a prime number}\}\) is the set that contains all and only the prime numbers. So we have that \(3\in \{x:x\text{ is a prime number}\}\) but \$4\textlnot{}\(\in\) $\backslash${x:x\text{ is a prime number}$\backslash$}.\$

Most formal languages that we will encounter will also be infinite sets. So what exactly \uline{is} a formal language? Put simply, a formal language is \textbf{\textbf{a set of sequences of symbols}}. A formal language starts with the specification of what these building blocks are. We call this an \textbf{\textbf{alphabet}}, which is just a set of symbols. Using the alphabet, we then use a \textbf{\textbf{grammar}} to construct the set sequences, i.e. the formal language.


\#\#\# Alphabets 

Sequences of symbols are recruited from an alphabet. We
usually write \(\Sigma\) to denote the alphabet of a language. 

It's important to note that the alphabet can be \uline{any} set. So, e.g., 

\(\Sigma=\{0,1,2,3,4,5,6,7,8,9\}\)

is a perfectly fine alphabet. You can use it to define the language for all the natural numbers. One way to do that is to use an operation called the Kleene star, named after the American mathematician [Stephen Kleene](\url{https://en.wikipedia.org/wiki/Stephen\_Cole\_Kleene}) and written as an asterix. The set $$\Sigma^*$$ is the set of all sequences that you can build with the elements of \(\Sigma\). This set is a formal language and it includes sequences such as \(15935304888\) and \(249583\) and \(2\). This is not the set of what we normally consider to be natural numbers, though, since \(\Sigma^*\) will also include sequences like \(0\), \(000000001\) and \(000881\), which are not natural numbers (in the Western Arabic decimal system, at least). So, while the Kleene star gives us a way to construct the set of all sequences made from an alphabet, most formal languages we are interested in will be a specific small subset of \(\Sigma^*\). This is why we need a \uline{grammar}.

\#\#\# Grammar 

The grammar of a language determines which strings of symbols from \(\Sigma\) are valid expressions of the language. 

In the case of most formal languages in logic, grammars use a technique known as
\textbf{\textbf{inductive definition}}. Here is an example of such a definition for the set of all numbers built from \(\Sigma\).

\begin{enumerate}
\item The following are all natural numbers: 1,2,3,4,5,6,7,8,9
\item If \(N\) is a natural number then so is \(N0\), \(N1\), \(N2\), \(N3\), \(N4\), \(N5\), \(N6\), \(N7\), \(N8\), \(N9\)
\item Nothing else is a natural number
\end{enumerate}

Here's how this definition works: In the first step we get all the natural numbers that can be written as a single digit. Then in the second step we can represent numbers that correspond to sequences of any length \(>1\). For instance, this definition shows that \(120\) is a natural number: (i) 1 is a natural number (step 1); (ii) 12 is a natural number (step 2); (iii) 120 is a natural number (step 2). Using this inductive definition, there is no way to show that \(01\) is a natural number. Given the final line of the definition, we must conclude that it is therefore not a natural number.

(The driving force behind this definition is actually an application of modus ponens. One instance of step 2 in the definition is: If \(1\) is a natural number, then so is \(12\). Now, since step 1 tells us that \(1\) is indeed a natural number it follows by modus ponens that \(12\) is also a natural number.) 

\#\#\# Application to logic

Just like the language of numeral notation we saw above, logics are also sets of sequences of symbols. We often refer to these sequences as \textbf{\textbf{formulas}}, so a logic is a formal language consisting of formulas. In order to specify such a language, we will want to specify an alphabet and a grammar so that the formulas that make up the formal language are well-formed sequences that are useful for the study of valid inference. Here, we will define the language used for \textbf{\textbf{propositional logic}}. 

Starting with the alphabet, we should first note that, in logic, not all elements of the alphabet play the same role. (Similarly, in the case of the language of numbers we saw that 0 played a different role than the other digits). For propositional logic, the alphabet consists of three kinds of things: 

\begin{itemize}
\item propositional variables: symbols that stand for propositions
\item operators: symbols that operate on or connect propositions
\item auxiliaries: symbols that indicate how parts of a formula combine
\end{itemize}

An example of an alphabet for the language of propositional logic is:


$$\Sigma_P = \{p_1,\ldots,p_n,\neg,\land,\lor,\rightarrow,\leftrightarrow,(,)\}$$

Here, \(p_1,\ldots p_n\) are the \uline{variables}, \(\neg\), \(\land\), \(\lor\), \(\rightarrow\), \(\leftrightarrow\) are the \uline{operators}, and \((\) and \()\) are the auxiliaries.

The Kleene star of this set, $$\Sigma^*$$ provides us with all the sequences that we can build using these symbols. $$\Sigma^*$$ contains meaningful expressions like:

$$((p_1\land p_3)\rightarrow \neg p_2)$$

but also lots of expressions that are not well-formed for propositional logic, like:

$$)p_1\neg\land((\rightarrow$$

So, we should give an inductive definition for the language of propositional logic, which we will call \(\mathcal{L}\):

\begin{itemize}
\item \(p_1,\dots, p_n \in \mathcal{L}\) and

\item if \(A\in\mathcal{L}\), then \(\neg A\in \mathcal{L}\) as well as

\item if \(A,B\in\mathcal{L}\), then \$(A\(\land\) B),(A\(\lor\) B),(A\(\rightarrow\) B),(A\(\leftrightarrow\)
\end{itemize}
B)\(\in\) \mathcal{L}\$

As before, crucially, we assume in addition that nothing else is in \(\mathcal{L}\), but from now on, we will leave this "closure condition" implicit. In other words, we assume that if something complies with the above statements, then it is indeed in \(\mathcal{L}\), but if it does not, then it is not. 

We can now easily see that \(((p_1\land p_2)\rightarrow \neg p_3))\) is a
member of \(\mathcal{L}\). To see this, we simply perform the construction:

\begin{enumerate}
\item We know that \(p_1\) is a formula.
\item We know that \(p_2\) is a formula.
\item We know that \(p_3\) is a formula.
\item Because of 3., we know that \(\neg p_3\) is a formula.
\item Because of 1. and 2., we know that \((p_1\land p_2)\) is a formula.
\item Because of 5. and 4., we know that \(((p_1\land p_2)\rightarrow \neg p_3)\) is a formula.
\end{enumerate}

But we can also see that \(\neg A\neg\) is not a formula, since no rule every
allows for \(\neg\) to occur in a formula without being followed by formula.

In computer science and AI, there is a wide-spread notation that significantly
simplifies the above rules: the so-called \textbf{\textbf{Backus-Naur Form (BNF)}}. In BNF,
instead of all of the above, we can simply write the following to define the same language \(\mathcal{L}\):

$$A::= p_i\mid\neg A\mid (A\land A)\mid (A\lor A)\mid (A\rightarrow A)\mid (A\leftrightarrow A)$$

Here, we read the "\(\mid\)" as an "or". And so this reads: a formula is either a
propositional variable, or the negation of a formula, or the conjunction of two
formulas, or \ldots{}.

You should know that BNFs sometimes take different forms. Here is an equivalent
way of giving the BNF for the same language:

$$\langle prop\rangle\mapsto p_1\mid \dots\mid p_n$$

$$\langle fml\rangle\mapsto\langle prop\rangle\mid\neg\langle fml\rangle\mid
(\langle fml\rangle\land \langle fml\rangle)\mid (\langle fml\rangle\lor
\langle fml\rangle)\mid $$
$$(\langle fml\rangle\rightarrow \langle fml\rangle)\mid (\langle
fml\rangle\leftrightarrow \langle fml\rangle)$$

but these are just
notational differences. 

[BNFs](\url{https://en.wikipedia.org/wiki/Backus\%E2\%80\%93Naur\_form}) are a \textbf{\textbf{powerful
method for defining formal languages}}. They are frequently used in logic,
computer science, and AI. For example, the syntax of most programming languages
is defined in BNF, see, e.g.,
[Python](\url{https://docs.python.org/3/reference/grammar.html}). Even if you want to
know what a valid email really is, you need to look up [its
BNF](\url{https://datatracker.ietf.org/doc/html/rfc5322}).

Importantly, the BNF we provided above is just one example of a logic that we can define on the basis of \(\Sigma_P\). One obvious way in which logics can differ is in the choice and number of propositional variables. More interestingly, some logics will only use a subset of the operators. For instance, the language \(\mathcal{L}'\) defined in the following way, is another example of a logical formal language. 

$$A::= p_i\mid\neg A\mid (A\land A)$$

In fact, it turns out that this language is equally \textbf{\textbf{expressive}} as the language \(\mathcal{L}\) above: everything that can be said using propositional variables and
    \(\neg,\land,\lor,\rightarrow,\leftrightarrow\) can also be said using just
    \(\neg,\land\). Seeing why is a topic for later in the course.

Translating natural language expressions into a formal language expression is
a process known as \textbf{\textbf{formalization}}. It's not always easy, but here are some
examples for formalization with propositional languages:

-------------------------------------------------------------------------- ------------ -----------------------------
The letter isn't in the left drawer                                         \(\leadsto\)            \(\neg p\)

It's not the case that the letter is in the left drawer                     \(\leadsto\)            \(\neg p\)

The letter is in the left and in the right drawer                           \(\leadsto\)            \((p\land q)\)

The letter is not in the left drawer, but it's also not in the right one    \(\leadsto\)     \((\neg p\land \neg q)\)

The letter is in the left or in the right drawer                            \(\leadsto\)           \((p\lor q)\)

The letter is neither in the left nor in the right drawer                   \(\leadsto\)     \((\neg p\land \neg q)\)

If the letter is in the left drawer, then it's not in the right drawer      \(\leadsto\)         \((p\rightarrow \neg q)\)

The letter is in the left drawer, if it's not in the right one              \(\leadsto\)         \((\neg q\rightarrow p)\)

The letter is only in the left drawer, if it's not in the right one         \(\leadsto\)         \((p\rightarrow \neg q)\)

The letter is in the left drawer just in case it's not in the right one     \(\leadsto\)   \((p\leftrightarrow \neg q)\)
-------------------------------------------------------------------------- ------------ -----------------------------

\textbf{\textbf{More examples}}

At this point, you know enough about how logical grammars and BNFs work that you
can check out your own examples. Here are some suggestions for grammars to check
out:


\begin{itemize}
\item Pick your favorite programming language (if you have one):
\end{itemize}
[Python](\url{https://docs.python.org/3/reference/grammar.html}) we mentioned above,
[C](\url{https://cs.wmich.edu/\~gupta/teaching/cs4850/sumII06/The\%20syntax\%20of\%20C\%20in\%20Backus-Naur\%20form.htm})
is a popular low-level language,
[Prolog](\url{http://tau-prolog.org/files/doc/grammar-specification.pdf}) is a
logic-based language. 

\begin{itemize}
\item A more complex AI example is [description
\end{itemize}
logic](\url{https://en.wikipedia.org/wiki/Description\_logic}), which is a powerful KR
language for designing knowledge bases.

\begin{itemize}
\item The [RFC](\url{https://datatracker.ietf.org/doc/html/rfc5322}) for emails contains
\end{itemize}
the BNF for valid email addresses. Check it out 🤓

\#\# Parsing

So far, we've looked at how to define formulas by looking at how they are
constructed from simpler formulas. Now, we'll invert the perspective and
\uline{desconstruct} or \textbf{\textbf{parse}} formulas.

\uline{Why?} you ask? Well, the reason we need to look into this is because that's
essentially what computers do to \textbf{\textbf{understand formulas}}. The idea of parsing is
to split a formula according to the rules of the grammar to recover how it was
constructed. By doing so we get insight into the syntactic structure of a formula. 

Consider the following BNF for a propositional logic with just three propositional 
constants \(p\), \(q\) and \(r\):

$$A::= p\mid q\mid r\mid\neg A\mid (A\land A)\mid (A\lor A)\mid (A\rightarrow A)\mid (A\leftrightarrow A)$$

Essentially, this is a collection of eight production rules:

--------  --------------------  -------
$$r_1$$:  $$A\Longrightarrow$$  $$p$$  
$$r_2$$:  $$A\Longrightarrow$$  $$q$$   
$$r_3$$:  $$A\Longrightarrow$$  $$r$$   
$$r_4$$:  $$A\Longrightarrow$$  $$\neg A$$  
$$r_5$$:  $$A\Longrightarrow$$  $$(A\land A)$$    
$$r_6$$:  $$A\Longrightarrow$$  $$(A\lor A)$$  
$$r_7$$:  $$A\Longrightarrow$$  $$(A\rightarrow A)$$  
$$r_8$$:  $$A\Longrightarrow$$  $$(A\leftrightarrow A)$$  
--------  --------------------  -------


Any formula in the language given by this BNF is a formula that can be \uline{derived} by applying a finite number of choices from these rules.
For instance, to show that \(\neg(p\land q)\) is a formula in this language, we start with \(A\) and, using the rules above, we rewrite this \(A\) until we arrive that this formula. We can do this in three steps:

$$A\ \Longrightarrow_{r_4} \neg A\ \Longrightarrow_{r_5} \neg (A\land A)\ \Longrightarrow_{r_1}\neg(p\land A)\ \Longrightarrow_{r_2}\neg(p\land q)$$

To make this more insightful, we can turn this derivation into a so-called \textbf{\textbf{parse tree}}, which is a very useful representation of the syntax of a formula. A parse tree is a structure that is rooted in the abstract label \(A\) that forms the base of our BNF definition of the language. You can construct a tree by just following the derivation above, step by step. Each application of a rule introduces a new branching, until there is nothing left doing anymore. Here is how to construct the parse tree for the derivation we gave for \(\neg(p\land q)\). We start with a node \(A\) and then look at the derivation to see which rule to apply first. This is rule 4, which maps \(A\) to a new formula \(\neg A\). For each new symbol we create a new branch:

<img src="tex\textsubscript{gen}/tree\textsubscript{1}\textsubscript{a.png}" alt="parse tree" width="100pt" class="img-thumbnail"> 

Then, we apply rule 5 to the right-most branch. This rule rewrites this \(A\) into \((A\land A)\), which creates five more branches:

<img src="tex\textsubscript{gen}/tree\textsubscript{1}\textsubscript{b.png}" alt="parse tree" width="200pt" class="img-thumbnail"> 

Then, we apply rule 1 to \(A\) that is to the left of "\(\land\)":

<img src="tex\textsubscript{gen}/tree\textsubscript{1}\textsubscript{c.png}" alt="parse tree" width="200pt" class="img-thumbnail" >

Finally, we apply rule 2:

<img src="tex\textsubscript{gen}/tree\textsubscript{1}\textsubscript{d.png}" alt="parse tree" width="200pt" class="img-thumbnail" >

This now is the parse tree corresponding to our derivation of \(\neg(p\land q)\). The leaves of the tree spell out the formula, each branching is an application of a rule from the BNF grammar.

Here is an example of a derivation for a more complicated formula:

$$A\ \Longrightarrow_{r_7} (A\rightarrow A)\ \Longrightarrow_{r_4} (A\rightarrow \neg A)\ \Longrightarrow_{r_2} (A\rightarrow \neg q)\ \Longrightarrow_{r_5} ((A\land A)\rightarrow \neg q)$$ 

$$\Longrightarrow_{r_1} ((p\land A)\rightarrow \neg q)\ \Longrightarrow_{r_7} ((p\land (A\rightarrow A))\rightarrow \neg q)$$ 

$$\Longrightarrow_{r_2} ((p\land (A\rightarrow q))\rightarrow\neg q)\ \Longrightarrow_{r_1} ((p\land (p\rightarrow q))\rightarrow\neg q)$$

The corresponding parse tree looks like this:

<img src="./tex\textsubscript{gen}/tree\textsubscript{2.png}" alt="parsing tree" width="300pt" class="img-thumbnail" >

A fundamental insight of logical theory is that when a grammar is properly
defined, we get what's known as \textbf{\textbf{unique readability}}. A formula has this property when the grammar only provides a single parse tree for it. This is the case for both examples we gave above. For instance, for \(((p\land (p\rightarrow q))\rightarrow\neg q)\) we don't have a choice in what rule to apply first when we start out derivation. We cannot for instance apply rule 4 before we apply rule 7. If we did, we would end up with a different formula. We have some choices later in the derivation. For instance, after applying rule 4, we could have chosen to apply rule 5 to the \(A\) to the left of the \(\rightarrow\). But that is not  a choice that affects the structure. The parse tree would remain the same. In other words, all derivation of this formula lead to the same tree.

Unique readability is of
the utmost importance since if it fails, this means that formulas are
\textbf{\textbf{ambiguous}}. Since one of the motivations for the use of formal languages is avoiding ambiguity, this means that we need to take special care in designing our grammar. Take the following logic, for instance:

$$ A:\ :=\ p \mid q \mid \neg A \mid A \land A$$

Using this grammar, we can derive $$\neg p\land q$$. Crucially, though, we can derive this in two distinct ways, corresponding to the following two parse trees.

<img src="tex\textsubscript{gen}/tree\textsubscript{3.png}" alt="parsing tree" width="300pt" class="img-thumbnail" >

Imagine we are building an AI system to regulate a train crossing. There is a light stopping traffic from crossing the railway when it turns red and similarly there is a light indicating the train should stop and wait with crossing the road until that light turns green. Let's say we have trained a neural network to regulate things as efficiently as possible, minimizing train delays and traffic jams. Unfortunately, the neural network is not flawless. We need a rule based system to make sure the decisions made by the network are safe. To do this, we translate the decisions to statements in a propositional logic and compare these to rules that we want the system to obey. Let's say that \(p\) means that the cars have a green light and \(q\) means that the train has a green light. We now want a rule that says that \(p\) and \(q\) cannot be true at the same time. 

As the two parse trees above show us, we have no way of doing this. If we state the rule as \(\neg p \land q\), we end up with something that could be misunderstood. The two trees correspond to two distinct derivations, which correspond to two different structures for the same formula. In turn this means that the formula will have two interpretations. On the right is the interpretation that would be handy for this AI system: cars and trains do not have a green light at the same time. But if the system instead adopts the structure on the left, we could end up with a system that demands that trains have a green light while cars do not. Ambiguity might just have created a huge traffic jam! This shows that the BNF above is unsuitable as a formal language, since it fails the property of unique readability.
All this is why we need to be careful about the \textbf{\textbf{auxiliaries}}, like \((,)\), which
ultimately guarantee unique readability. 

Parsing is an incredibly important subject in the foundations and practice of
programming, natural language processing (NLP), and elsewhere. We don't have
time to go into the details, but think of programming for a second. A
[programming language](\url{https://en.wikipedia.org/wiki/Programming\_language}) is
essentially a tool to write down instructions for a computer in a human-readable
way. What happens "under the hood" is that the computer translates the program
you write into machine instructions (the proverbial 1's and 0's). 

To ensure that the machine instructions really correspond to what you had in
mind when you wrote the program, the computer needs to understand \uline{what you
meant}. Since a computer is deterministic and not \uline{particularly} intelligent,
the only way it can do this is according to clear instructions about what means
what. 

But clearly, we can't just write for each program what it means in machine
instructions-otherwise, what's the point of having the language in the first
place? Instead, we specify what the individual expressions of the language mean
and how combining them according to the grammar affects that meaning. In this
way, we guarantee that for each program we could possibly write, we can
translate it into machine instructions. 

But to do so, we need to know which expressions occur in which order in the
program. To determine this is the role of the
[parser](\url{https://en.wikipedia.org/wiki/Parsing\#Parser}). This shows the
fundamental importance of parsing in programming and human-computer interaction.

\#\# Knowledge representation

One of the most important applications of formal languages is as a formalism that allows storing knowledge. This could be knowledge of any kind: facts about the world, facts about the clients and their orders of a major company, the location of objects in a space that a robot is navigating, etc.

Above we introduced a formal language for propositional logic as follows:

$$A::= p_i\mid\neg A\mid (A\land A)\mid (A\lor A)\mid (A\rightarrow A)\mid (A\leftrightarrow A)$$

We could use this for knowledge representation, but it will be easy to see that its applications will be rather limited. Say, we want to represent the knowledge we have about the location of three objects, \(A\), \(B\) and \(C\). We know that \(A\) is left of \(B\) and \(B\) is on left of \(C\). How do we store that knowledge using propositional logic. Unfortunately, all we can do is something like this: First we state that \(p_1\) corresponds to \(A\) is left of \(B\) and that \(p_2\) corresponds to \(B\) is left of \(C\). Then, we declare all our knowledge:

$$p_1$$

$$p_2$$

This is hardly useful. What we see here is that although propositional logic is unambiguous and really useful for stating general cases of valid inference, it is \textbf{\textbf{under-expressive}} for most applications that concern knowledge representation. Propositional logic can only represent propositions, things that are true or false. These propositions themselves have no aboutness and, in particular, there is no role in propositional logic for (representations of) objects in the world. In other words, we would like to have a formal language that not just state that certain things are true or false, but be explicit about what these propositions are about. In \textbf{\textbf{first order predicate logic}}, for instance, we can express the location of the objects \(A\), \(B\) and \(C\) much more transparently:

$$left(A,B)$$

$$left(B,C)$$

Here, "left" is a predicate and it expresses a relation that holds of two entities. Additionally, using first order logic, we can express relations that hold in general. Consider, for instance, the following rule:

$$\forall x\forall y\forall z[(left(x,y)\land left(y,z)\rightarrow left(x,z)]$$

This rule states that if for any object \(x\), for any object \(y\) and for any object \(z\) it holds that if \(x\) is to the left of \(y\) and \(y\) is to the left of \(z\), then it follows that \(x\) is to the left of \(z\). Using this rule and the two facts above, we can derive a new fact, namely that \(left(A,C)\).

\textbf{\textbf{First-order logic (FOL)}} is extremely important in logical theory and in AI
applications. In part, this is because FOL has a lot of \textbf{\textbf{expressive power}}: a
lot of
claims—[some](\url{https://en.wikipedia.org/wiki/John\_McCarthy\_(computer\_scientist)})
would say \_everything\_—can be formalized in it. 

Let us turn to defining FOL as a formal language. In general, the alphabet looks something like this:

$$\{a,b,c,\dots,
x,y,z,\dots,f,g,h,\dots,P,Q,R,\dots,\neg,\land,\lor,\rightarrow,\leftrightarrow,\forall,\exists,(,),,\}$$

\textbf{\textbf{NB}}: There's no typo at the end here 😃 The last symbol is a literal comma.

For a concrete language, we'd need to pick some suitable constants
\(a,b,c,\dots\), function symbols \(f,g,h,\dots\), and predicates \(P,Q,R,\dots\).
Usually, in KR-contexts, these will be mnemonic, of course. 

The full syntax of FOL, then, is:

$$\langle const\rangle ::= a \mid b\mid \dots $$
$$\langle var\rangle ::= x \mid y\mid \dots $$
$$\langle unop\rangle ::= \neg$$
$$\langle binop\rangle ::= \land\mid\lor\mid\rightarrow\mid\leftrightarrow$$
$$\langle quant\rangle ::= \forall\mid\exists$$
$$\langle fu n^n\rangle ::= f^n\mid g^n\mid \dots $$
$$\langle term\rangle::= \langle const\rangle\mid\langle variable\rangle\mid
\langle fun^n\rangle(\overbrace{\langle term\rangle,\dots,\langle term\rangle}^{n\text{ times}})$$
$$\langle pred^n\rangle ::= P^n\mid Q^n\mid \dots $$
$$\langle atom\rangle::= \langle pred^n\rangle(\underbrace{\langle term\rangle,\dots\langle term\rangle}_{n\text{ times}})$$
$$\langle fml\rangle::=\langle atom\rangle\mid\langle unop\rangle\langle fml\rangle\mid
(\langle fml\rangle\langle binop\rangle \langle fml\rangle)\mid \langle quant\rangle \langle
var\rangle\langle fml\rangle$$

As you can see, the syntax of FOL is significantly more complex than the syntax
of propositional logic. But syntactically, nothing too complicated is going on.
Assuming, for example, that \(FRIEND\) is a binary predicate and \(data\) a constant
for data in our language, we can write:

$$\exists xFRIEND(data,x)$$

to say that Data has a friend.

Here are some more suggestions on how to formalize:

------------------------------------------- ----------- -------------------------------
Not everybody handsome is smart             \(\leadsto\)  \(\neg\forall x(H(x)\rightarrow S(x))\)

Everybody who's smart is handsome           \(\leadsto\)  \(\forall x(S(x)\rightarrow H(x))\)

A person who's smart is handsome            \(\leadsto\)  \(\forall x(S(x)\rightarrow H(x))\)

Someone who's smart is handsome             \(\leadsto\)  \(\forall x(S(x)\rightarrow H(x))\)

Everybody's smart and handsome              \(\leadsto\)  \(\forall x(S(x)\land H(x))\)

Somebody who's smart exists                 \(\leadsto\)  \(\exists x S(x)\)

There's somebody who's not smart            \(\leadsto\)  \(\exists x\neg S(x)\)

Somebody's smart and somebody's handsome    \(\leadsto\)  \(\exists xS(x)\land \exists xH(x)\)

Somebody's smart and handsome               \(\leadsto\)  \(\exists x(S(x)\land H(x))\)

Nobody's both smart and handsome            \(\leadsto\)  \(\neg\exists x(S(x)\land H(x))\)

Somebody, who's smart, is handsome          \(\leadsto\)  \(\exists x(S(x)\land H(x))\)

------------------------------------------- ----------- -------------------------------

Indeterminate terms, like pronouns, indexicals, etc., are formalized using variables. Only when clearly the same thing is meant, use the same variable, if different things could be meant, use different variables:

------------------------------------ ------------ ------------------
He's handsome                         \(\leadsto\)  \(H(x)\)

She's handsome and smart              \(\leadsto\)  \(H(x)\land S(x)\)

He's handsome and \textbf{he}'s smart        \(\leadsto\)  \(H(x)\land S(y)\)

He's handsome and she's smart         \(\leadsto\)  \(H(x)\land S(y)\)

That's a smart and handsome person    \(\leadsto\)  \(H(x)\land S(x)\)
------------------------------------ ------------ ------------------

Most languages used for KR are \textbf{\textbf{fragments}} of FOL, since FOL has certain
theoretical limitations, which we'll discuss later in the course.

\#\# Applications

Let's talk for a moment about the role of formal languages in AI. We've already
talked about the fact that programming languages are essentially just formal
languages. So we can use the theory of formal languages to understand this
aspect of human-machine interaction, which is crucial also in AI. Here are some
other applications:

\#\#\# Natural language processing

One application of formula languages, parsing, and related techniques is in
[**natural language processing
(NLP)**](\url{https://en.wikipedia.org/wiki/Natural\_language\_processing}). For a long
time (until roughly the 1990s), formal languages played a key role in NLP in
what in analogy to symbolic AI is known as [symbolic
NLP](\url{https://en.wikipedia.org/wiki/Natural\_language\_processing\#Symbolic\_NLP\_(1950s\_\%E2\%80\%93\_early\_1990s)}).

As suggested by Wikipedia, we can use a famous thought experiment known as the
\uline{Chinese room} to illustrate the idea:

\{\{< blockquote author="Jon Searle. 1999. 'The Chinese Room'">\}\}
Imagine a native English speaker who knows no Chinese locked in a room full of boxes of Chinese symbols (a data base) together with a book of instructions for manipulating the symbols (the program). Imagine that people outside the room send in other Chinese symbols which, unknown to the person in the room, are questions in Chinese (the input). And imagine that by following the instructions in the program the man in the room is able to pass out Chinese symbols which are correct answers to the questions (the output). The program enables the person in the room to pass the Turing Test for understanding Chinese but he does not understand a word of Chinese. 
\{\{< /blockquote >\}\}
Essentially, the computer is the person in the room. The rules are parsing rules
and formal grammars, which enable the room to "speak Chinese".

The idea was to build NLP technologies in a similar way and for a while this was
moderately successful. But much for the same reasons why symbolic AI in general
"failed", symbolic NLP is   no longer a strong paradigm in NLP.  While symbolic
methods are still around, in NLP, statistical methods, which are at the core of
[LLMs](\url{https://en.wikipedia.org/wiki/Large\_language\_model}), for example, rule
the waves.


\#\#\# Knowledge representation

Things are more interesting when it comes to KR. To this day, formal languages
and knowledge bases (KBs) are \textbf{\textbf{powerful tools}} when it comes to storing and
making accessible known facts to computational systems, such as computers or
AI-systems. Mathematically speaking, a knowledge bases is just a set of
formulas. In a slogan: $$\mathbf{KB}\subseteq \mathcal{L}$$ 

The main strengths of KBs is their \textbf{\textbf{reliability}} and \textbf{\textbf{precision}}. Mistakes
in KBs are essentially only due to human error. Interestingly, though, there is
[ongoing research](\url{https://arxiv.org/abs/2407.13578}) on using LLMs, for example,
to store factual information, even though they can't compete with knowledge
bases yet.

What we should note, though, is that the formal languages that are used for KR
are usually \textbf{\textbf{less expressive}} than FOL. This is due to some theoretical
results about FOL, which provide fundamental roadblocks to using its full
expressive power in computational contexts. We've already briefly touched upon
one such reason in \{\{< chapter\textsubscript{ref} chapter="logic-and-ai"
id="as-a-foundation">\}\} Chapter 1. Logic and AI\{\{< /chapter\textsubscript{ref} >\}\}, when we
spoke about Turing's [undecidability
theorem](\url{https://en.wikipedia.org/wiki/Decidability\_\%28logic\%29}), which states
that validity checking in FOL, specifically, cannot be fully automated.
[Description logic](\url{https://en.wikipedia.org/wiki/Description\_logic}) is an
interesting example of an approach to KR that uses what's effectively a fragment
of FOL KR-purposes.

An active area of research is the so-called [semantic
web](\url{https://en.wikipedia.org/wiki/Semantic\_Web}), which uses languages like
[OWL](\url{https://en.wikipedia.org/wiki/Web\_Ontology\_Language}) to make data on the
internet machine readable. 

\#\# Further readings

An incredibly rich and extensive discussion of formal languages and their role
in logic is:

\begin{itemize}
\item\relax [Duthil Novaes, Catarina. 2012. Formal languages in logic. Cambridge
\end{itemize}
University Press](\url{https://doi.org/10.1017/CBO9781139108010})

From a linguistic perspective, a highly influential idea is Montague's idea to
understand "English as a formal language":

\begin{itemize}
\item\relax [Montague, Richard. 1968.English as a formal
\end{itemize}
language.](\url{https://doi.org/10.1515/9783111546216-007})

\textbf{\textbf{Notes:}}

[\textsuperscript{history}]: See the book by Duthil Novaes, for example.
There is not so much more to be said about the alphabet but it's useful to
remark that in logical contexts, there are some special kinds of symbols that
are usually used in the alphabets, which have special meanings.

Here is a (non-exhaustive) list. Typically, we distinguish between:

\textbf{\textbf{Non-logical symbols}}

These symbols are typically the result of logical abstraction. But in knowledge
representation contexts, they can also be the result of representing
extra-logical information.

\begin{enumerate}
\item \textbf{\textbf{Propositional variables}} also known as \textbf{\textbf{sentence letters}}. 

These stand for sentences, like "it is raining" or "logic is awesome". When
it doesn't matter which sentences we're talking about, they are often
\(p,q,r,\dots\). In knowledge representation (KR) contexts, they can also be
mnemonic, like \(RAIN\) or \(AWESOME\).

\item \textbf{\textbf{Constants}}. 

These stand for proper names, like "Alan" or "Ada". In logic, they are often
\(a,b,c,\dots\), but in KR-contexts, they can also be mnemonic, like \(alan\) or
\(ada\). Sometimes, they are just ordinary numerals, like \(0,1,2,\dots\) or
\(\pi\).
\end{enumerate}


\begin{enumerate}
\item \textbf{\textbf{Function symbols}}. 

These stand for [functional
expressions](\url{https://en.wikipedia.org/wiki/Function\_(mathematics)}), like "+"
or "the father of". In logic, often \(f,g,h,\dots\) and in KR often mnemonic,
like \(FatherOf\). In mathematical logic, function symbols like \(+,-,\cdot,
    \dots\) are common, too.

\item \textbf{\textbf{Predicates}}. 

They stand for \ldots{}
[predicates](\url{https://en.wikipedia.org/wiki/Predicate\_(grammar)https://en.wikipedia.org/wiki/Predicate\_(grammar)}),
which are expressions that define properties or relationships, like "being
blue" or "being greater than". In logic, usually \(P,Q,R,\dots\) and in KR,
also mnemonics, like \(BLUE\) or \(GREATER\).

A special case is the identity symbol \(=\), which some logicians treat as
logical and some as non-logical. Otherwise, it works just like predicate.
\end{enumerate}

\textbf{\textbf{NB}}: Function symbols and predicates come with an \uline{arity}, which is how many
arguments they take. In syntax specifications, we often write this as a
superscript. So, for example, the fact that \(BLUE\) applies to one thing (it's
\uline{unary}) would be written \(BLUE^1\). 

\textbf{\textbf{Logical symbols}}

These are the result of idealization. Which logical symbols are available
depends on the logical system. There symbols for \uline{many} logically relevant
concepts.

\begin{enumerate}
\item \textbf{\textbf{Variables}}.

These stand for arbitrary but concrete individuals or properties. They have
a mainly logical function in the context of quantification, which we'll
cover more extensively later in the book.

They are typically \(x,y,z,\dots\) but sometimes \(\alpha,\beta,\delta\), when
we're talking about individuals. And they are typically \(X,Y,Z,\dots\) when
we're talking about variables for properties.
\end{enumerate}


\begin{enumerate}
\item \textbf{\textbf{Sentential operators}}.

These connect one or more sentences or phrases to form a new one. Typical
examples are the \textbf{\textbf{classical propositional connectives}}:

\begin{center}
\begin{tabular}{lll}
\textbf{\textbf{Symbol}} & \&nbsp; & \textbf{\textbf{Meaning}}\\
------------ & -- & ---------\\
\(\neg,\sim\) &  & not\\
\(\land,\\&\) &  & and\\
\(\lor\) &  & or\\
\(\rightarrow,\Rightarrow,\supset\) &  & if \ldots{}, then \ldots{}\\
\(\leftrightarrow,\Leftrightarrow,\equiv\) &  & iff\\
\(\vdots\) &  & \(\vdots\)\\
\end{tabular}
\end{center}

But many other operators are known and/or can be introduced:

\begin{center}
\begin{tabular}{lll}
\textbf{\textbf{Symbol}} & \&nbsp;\&nbsp; & \textbf{\textbf{Meaning}}\\
------------ & -- & ---------\\
\(\square,\lozenge\) &  & necessity, possibility\\
\(K,B\) &  & knowledge, belief\\
\(G,F,H,P\) &  & past, future\\
\(P,O\) &  & permission, obligation\\
\(!\) &  & announcement\\
\(?\) &  & questions\\
\(\vdots\) &  & \(\vdots\)\\
\end{tabular}
\end{center}

Unfortunately, we won't be able to cover most of these more advanced
operators in detail.

The operators listed above are standard \uline{logical} operators. But note that
in programming languages, for example, we often have mnemonic conditionals,
like in the following [pseudocode](\url{https://en.wikipedia.org/wiki/Pseudocode}), for example:

```
  IF \ldots{} THEN
      \ldots{}
  ELSE
      \ldots{}
  END IF
```

Similarly, programming languages often have idiosyncratic notation for the
classical propositional connectives (which are, of course, easier to type on
ordinary keyboards), such as \(||\) for disjunction, \(\\&\\&\) for conjunction
in [C](\url{https://en.wikipedia.org/wiki/C\_(programming\_language)}), or simply
\(\mathsf{and}\), \(\mathsf{or}\), \(\mathsf{not}\) in
[Pyton](\url{https://en.wikipedia.org/wiki/Python\_(programming\_language)}).

\item \textbf{\textbf{Quantifiers}}.

These allow us to express claims about all (\(\forall\)) or some (\(\exists\))
things. More generally, quantifiers allow us to make \uline{generalizations}.
There are also specialized quantifiers, such as numeric quantifiers, like
\(\exists_3\) which says "there are exactly 3".
\end{enumerate}

\textbf{\textbf{Auxiliaries}}.

\begin{enumerate}
\item \textbf{\textbf{Parsing}} 

These are symbols that help the notation of the language. They are things
like commas "," or parentheses "(" and ")". They don't have a meaning
themselves, but they help us to disambiguate formulas. They are important
for parsing (see below).
\end{enumerate}

These are, in any case, only examples of some common symbols in the alphabets of
formal languages. Ultimately, the sky is the limit.

The formulas, then, are sequences of symbols from the alphabet. But not every
sequence of symbols is a formula, formulas are constructed from the symbols
according to \uline{rules}.

The formal languages we use in logic and KR are usually rather simple in that
they allow for uncomplicated grammars. The complex grammatical phenomena we
often encounter in natural languages, for example, which are required for to
capture all linguistic nuances (which are often \uline{logically} irrelevant), we need
more sophisticated grammas, like [context-sensitive
grammars](\url{https://en.wikipedia.org/wiki/Context-sensitive\_grammar}).




\#\# Ideas

Like much of logic, formal languages have a \uline{long} history.[\textsuperscript{history}] The use of
abstract sentence letters as in $$A\text{ and }B$$ to express logical form can
already be found in [Aristotle's
\_Organon\_](\url{https://en.wikipedia.org/wiki/Organon}). Leibniz's [characteristica
universalis](\url{https://en.wikipedia.org/wiki/Characteristica\_universalis}) is
perhaps the first attempt at defining a formal language in the modern sense. In
mathematics, the rise of formal languages is associated with rise of logical
rigor in the foundations of mathematics, culminating in
[logicist](\url{https://en.wikipedia.org/wiki/Logicism}) projects, such as Gottlob
Frege's [Begriffsschrift](\url{https://en.wikipedia.org/wiki/Begriffsschrift}).
\end{document}